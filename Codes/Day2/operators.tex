% Options for packages loaded elsewhere
\PassOptionsToPackage{unicode}{hyperref}
\PassOptionsToPackage{hyphens}{url}
%
\documentclass[
]{article}
\usepackage{amsmath,amssymb}
\usepackage{iftex}
\ifPDFTeX
  \usepackage[T1]{fontenc}
  \usepackage[utf8]{inputenc}
  \usepackage{textcomp} % provide euro and other symbols
\else % if luatex or xetex
  \usepackage{unicode-math} % this also loads fontspec
  \defaultfontfeatures{Scale=MatchLowercase}
  \defaultfontfeatures[\rmfamily]{Ligatures=TeX,Scale=1}
\fi
\usepackage{lmodern}
\ifPDFTeX\else
  % xetex/luatex font selection
\fi
% Use upquote if available, for straight quotes in verbatim environments
\IfFileExists{upquote.sty}{\usepackage{upquote}}{}
\IfFileExists{microtype.sty}{% use microtype if available
  \usepackage[]{microtype}
  \UseMicrotypeSet[protrusion]{basicmath} % disable protrusion for tt fonts
}{}
\makeatletter
\@ifundefined{KOMAClassName}{% if non-KOMA class
  \IfFileExists{parskip.sty}{%
    \usepackage{parskip}
  }{% else
    \setlength{\parindent}{0pt}
    \setlength{\parskip}{6pt plus 2pt minus 1pt}}
}{% if KOMA class
  \KOMAoptions{parskip=half}}
\makeatother
\usepackage{xcolor}
\usepackage[margin=1in]{geometry}
\usepackage{color}
\usepackage{fancyvrb}
\newcommand{\VerbBar}{|}
\newcommand{\VERB}{\Verb[commandchars=\\\{\}]}
\DefineVerbatimEnvironment{Highlighting}{Verbatim}{commandchars=\\\{\}}
% Add ',fontsize=\small' for more characters per line
\usepackage{framed}
\definecolor{shadecolor}{RGB}{248,248,248}
\newenvironment{Shaded}{\begin{snugshade}}{\end{snugshade}}
\newcommand{\AlertTok}[1]{\textcolor[rgb]{0.94,0.16,0.16}{#1}}
\newcommand{\AnnotationTok}[1]{\textcolor[rgb]{0.56,0.35,0.01}{\textbf{\textit{#1}}}}
\newcommand{\AttributeTok}[1]{\textcolor[rgb]{0.13,0.29,0.53}{#1}}
\newcommand{\BaseNTok}[1]{\textcolor[rgb]{0.00,0.00,0.81}{#1}}
\newcommand{\BuiltInTok}[1]{#1}
\newcommand{\CharTok}[1]{\textcolor[rgb]{0.31,0.60,0.02}{#1}}
\newcommand{\CommentTok}[1]{\textcolor[rgb]{0.56,0.35,0.01}{\textit{#1}}}
\newcommand{\CommentVarTok}[1]{\textcolor[rgb]{0.56,0.35,0.01}{\textbf{\textit{#1}}}}
\newcommand{\ConstantTok}[1]{\textcolor[rgb]{0.56,0.35,0.01}{#1}}
\newcommand{\ControlFlowTok}[1]{\textcolor[rgb]{0.13,0.29,0.53}{\textbf{#1}}}
\newcommand{\DataTypeTok}[1]{\textcolor[rgb]{0.13,0.29,0.53}{#1}}
\newcommand{\DecValTok}[1]{\textcolor[rgb]{0.00,0.00,0.81}{#1}}
\newcommand{\DocumentationTok}[1]{\textcolor[rgb]{0.56,0.35,0.01}{\textbf{\textit{#1}}}}
\newcommand{\ErrorTok}[1]{\textcolor[rgb]{0.64,0.00,0.00}{\textbf{#1}}}
\newcommand{\ExtensionTok}[1]{#1}
\newcommand{\FloatTok}[1]{\textcolor[rgb]{0.00,0.00,0.81}{#1}}
\newcommand{\FunctionTok}[1]{\textcolor[rgb]{0.13,0.29,0.53}{\textbf{#1}}}
\newcommand{\ImportTok}[1]{#1}
\newcommand{\InformationTok}[1]{\textcolor[rgb]{0.56,0.35,0.01}{\textbf{\textit{#1}}}}
\newcommand{\KeywordTok}[1]{\textcolor[rgb]{0.13,0.29,0.53}{\textbf{#1}}}
\newcommand{\NormalTok}[1]{#1}
\newcommand{\OperatorTok}[1]{\textcolor[rgb]{0.81,0.36,0.00}{\textbf{#1}}}
\newcommand{\OtherTok}[1]{\textcolor[rgb]{0.56,0.35,0.01}{#1}}
\newcommand{\PreprocessorTok}[1]{\textcolor[rgb]{0.56,0.35,0.01}{\textit{#1}}}
\newcommand{\RegionMarkerTok}[1]{#1}
\newcommand{\SpecialCharTok}[1]{\textcolor[rgb]{0.81,0.36,0.00}{\textbf{#1}}}
\newcommand{\SpecialStringTok}[1]{\textcolor[rgb]{0.31,0.60,0.02}{#1}}
\newcommand{\StringTok}[1]{\textcolor[rgb]{0.31,0.60,0.02}{#1}}
\newcommand{\VariableTok}[1]{\textcolor[rgb]{0.00,0.00,0.00}{#1}}
\newcommand{\VerbatimStringTok}[1]{\textcolor[rgb]{0.31,0.60,0.02}{#1}}
\newcommand{\WarningTok}[1]{\textcolor[rgb]{0.56,0.35,0.01}{\textbf{\textit{#1}}}}
\usepackage{graphicx}
\makeatletter
\def\maxwidth{\ifdim\Gin@nat@width>\linewidth\linewidth\else\Gin@nat@width\fi}
\def\maxheight{\ifdim\Gin@nat@height>\textheight\textheight\else\Gin@nat@height\fi}
\makeatother
% Scale images if necessary, so that they will not overflow the page
% margins by default, and it is still possible to overwrite the defaults
% using explicit options in \includegraphics[width, height, ...]{}
\setkeys{Gin}{width=\maxwidth,height=\maxheight,keepaspectratio}
% Set default figure placement to htbp
\makeatletter
\def\fps@figure{htbp}
\makeatother
\setlength{\emergencystretch}{3em} % prevent overfull lines
\providecommand{\tightlist}{%
  \setlength{\itemsep}{0pt}\setlength{\parskip}{0pt}}
\setcounter{secnumdepth}{-\maxdimen} % remove section numbering
\ifLuaTeX
  \usepackage{selnolig}  % disable illegal ligatures
\fi
\usepackage{bookmark}
\IfFileExists{xurl.sty}{\usepackage{xurl}}{} % add URL line breaks if available
\urlstyle{same}
\hypersetup{
  hidelinks,
  pdfcreator={LaTeX via pandoc}}

\author{}
\date{\vspace{-2.5em}}

\begin{document}

\begin{Shaded}
\begin{Highlighting}[]

\FunctionTok{\#\#\# Summary of Operators in R}

\FunctionTok{\#\#\#\# 1. **Relational Operators**:}
\NormalTok{These operators help in comparing values and return a logical (Boolean) result (TRUE or FALSE).}

\SpecialStringTok{{-} }\InformationTok{\textasciigrave{}==\textasciigrave{}}\NormalTok{: Checks equality.}
\SpecialStringTok{{-} }\InformationTok{\textasciigrave{}!=\textasciigrave{}}\NormalTok{: Checks inequality.}
\SpecialStringTok{{-} }\InformationTok{\textasciigrave{}\textgreater{}\textasciigrave{}}\NormalTok{: Checks if the left value is greater than the right.}
\SpecialStringTok{{-} }\InformationTok{\textasciigrave{}\textgreater{}=\textasciigrave{}}\NormalTok{: Checks if the left value is greater than or equal to the right.}
\SpecialStringTok{{-} }\InformationTok{\textasciigrave{}\textless{}\textasciigrave{}}\NormalTok{: Checks if the left value is less than the right.}
\SpecialStringTok{{-} }\InformationTok{\textasciigrave{}\textless{}=\textasciigrave{}}\NormalTok{: Checks if the left value is less than or equal to the right.}

\NormalTok{Example:}

\InformationTok{\textasciigrave{}\textasciigrave{}\textasciigrave{} r}
\NormalTok{var\_1 }\OtherTok{=} \FunctionTok{c}\NormalTok{(}\DecValTok{1}\NormalTok{, }\DecValTok{2}\NormalTok{, }\DecValTok{3}\NormalTok{, }\DecValTok{10}\NormalTok{, }\DecValTok{11}\NormalTok{)}
\NormalTok{var\_2 }\OtherTok{=} \DecValTok{10}
\NormalTok{var\_1 }\SpecialCharTok{==}\NormalTok{ var\_2   }\CommentTok{\# FALSE FALSE FALSE TRUE FALSE}
\end{Highlighting}
\end{Shaded}

\begin{verbatim}
## [1] FALSE FALSE FALSE  TRUE FALSE
\end{verbatim}

\begin{Shaded}
\begin{Highlighting}[]
\NormalTok{var\_1[var\_1 }\SpecialCharTok{==}\NormalTok{ var\_2]  }\CommentTok{\# 10}
\end{Highlighting}
\end{Shaded}

\begin{verbatim}
## [1] 10
\end{verbatim}

\paragraph{\texorpdfstring{2. \textbf{Logical
Operators}:}{2. Logical Operators:}}\label{logical-operators}

These operators combine multiple logical conditions and return logical
values.

\begin{itemize}
\tightlist
\item
  \texttt{\&}: Logical AND.
\item
  \texttt{\textbar{}}: Logical OR.
\item
  \texttt{!}: Logical NOT.
\end{itemize}

Example:

\begin{Shaded}
\begin{Highlighting}[]
\NormalTok{var\_3 }\OtherTok{\textless{}{-}} \DecValTok{10}
\NormalTok{var\_3 }\SpecialCharTok{==}\NormalTok{ var\_2 }\SpecialCharTok{\&}\NormalTok{ var\_3 }\SpecialCharTok{\textgreater{}=}\NormalTok{ var\_2  }\CommentTok{\# TRUE}
\end{Highlighting}
\end{Shaded}

\begin{verbatim}
## [1] TRUE
\end{verbatim}

\begin{Shaded}
\begin{Highlighting}[]
\NormalTok{var\_3 }\SpecialCharTok{==}\NormalTok{ var\_2 }\SpecialCharTok{\&}\NormalTok{ var\_3 }\SpecialCharTok{\textgreater{}}\NormalTok{ var\_2   }\CommentTok{\# FALSE}
\end{Highlighting}
\end{Shaded}

\begin{verbatim}
## [1] FALSE
\end{verbatim}

\paragraph{\texorpdfstring{3. \textbf{Membership
Operator}:}{3. Membership Operator:}}\label{membership-operator}

Checks if a value is present in a vector or not.

\begin{itemize}
\tightlist
\item
  \texttt{\%in\%}: Checks if an element exists in a vector.
\end{itemize}

Example:

\begin{Shaded}
\begin{Highlighting}[]
\NormalTok{vector\_1 }\OtherTok{\textless{}{-}} \FunctionTok{c}\NormalTok{(}\DecValTok{1}\SpecialCharTok{:}\DecValTok{5}\NormalTok{)}
\NormalTok{value\_a }\OtherTok{\textless{}{-}} \DecValTok{4}
\NormalTok{value\_b }\OtherTok{\textless{}{-}} \DecValTok{6}
\NormalTok{value\_a }\SpecialCharTok{\%in\%}\NormalTok{ vector\_1  }\CommentTok{\# TRUE}
\end{Highlighting}
\end{Shaded}

\begin{verbatim}
## [1] TRUE
\end{verbatim}

\begin{Shaded}
\begin{Highlighting}[]
\NormalTok{value\_b }\SpecialCharTok{\%in\%}\NormalTok{ vector\_1  }\CommentTok{\# FALSE}
\end{Highlighting}
\end{Shaded}

\begin{verbatim}
## [1] FALSE
\end{verbatim}

These operators are fundamental for making decisions and performing
conditional operations in R. ```

When you save and render this R Markdown notebook, it will produce an
HTML document containing the summary and code examples along with any
plots or outputs.

\end{document}
